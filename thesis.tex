%@+leo-ver=5-thin
%@+node:gcross.20110314174620.1274: * @file thesis.tex
%@@language latex

%@+<< Prelude >>
%@+node:gcross.20110314174620.1275: ** << Prelude >>
\documentclass{book}

\usepackage{cite}
%@-<< Prelude >>

\begin{document}

%@+others
%@+node:gcross.20110314174620.1276: ** Introduction
\chapter{Introduction}

When physicists discovered the laws of quantum mechanics, they were both excited and disappointed.  On the one hand, this new theory did a fantastic job of modeling all of the bizarre microscopic phenomena that they had been encountering in ther labs.  But on the other hand, the theory was fundamentally \emph{nondeterministic}, postulating that reality was generally not in a single observable state but rather existed in many states at once; upon measurement one is selected at random and the rest discarded.  This made many physcists uncomfortable, as most famously expressed by Einstein:

\begin{quote}
Quantum mechanics is certainly imposing. But an inner voice tells me that it is not yet the real thing. The theory says a lot, but does not really bring us any closer to the secret of the `old one.' I, at any rate, am convinced that \emph{He is} not playing at dice.~\cite{Born2004}
\end{quote}
%@+node:gcross.20110316123044.1278: ** Bibliography
\bibliography{thesis}
\bibliographystyle{plain}
%@-others

\end{document}
%@-leo
